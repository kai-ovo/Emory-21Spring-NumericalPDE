\documentclass{article}
\usepackage[english]{babel}
\usepackage{geometry,amsmath,amssymb,enumerate,theorem}
\geometry{letterpaper}

%%%%%%%%%% Start TeXmacs macros
\newcommand{\tmem}[1]{{\em #1\/}}
\newcommand{\tmmathbf}[1]{\ensuremath{\boldsymbol{#1}}}
\newcommand{\tmname}[1]{\textsc{#1}}
\newcommand{\tmop}[1]{\ensuremath{\operatorname{#1}}}
\newcommand{\tmsamp}[1]{\textsf{#1}}
\newcommand{\tmstrong}[1]{\textbf{#1}}
\newcommand{\tmtt}[1]{\texttt{#1}}
\newenvironment{enumeratenumeric}{\begin{enumerate}[1.] }{\end{enumerate}}
{\theorembodyfont{\rmfamily\small}\newtheorem{problem}{Problem}}
%%%%%%%%%% End TeXmacs macros

\begin{document}

\title{MATH352 - FINAL PROJECT}

\maketitle

{\tmsamp{INSTRUCTIONS:
Please, refer to the Syllabus for Instructions.
Report the code you use in the Appendix (if you are not using Jupyter, in that
case, the code is in the body), report the results (pictures, tables, etc.) in
the body of the report.
No page limit, but synthesis is appreciated.}}

{\tmsamp{THIS IS AN INDIVIDUAL WORK}}. {\tmsamp{ANY RESOURCE (INCLUDING
ON-LINE) BEYOND THE MATERIAL ON CANVAS MUST BE CITED IN THE REPORT.}}

\begin{problem}
  In this first part, we consider the Black and Scholes equation - Put Option
  (see Hands On session, week 7): $V$ is the value of the Put option based on
  the asset with price {\tmname{$x$}}, $E$ is the strike price, $r$ the
  inflaction rate and $\sigma$ the volatility: \
  \[ \left\{\begin{array}{l}
       \dfrac{\partial V}{\partial \tau} + \dfrac{1}{2} \sigma^2 x^2
       \dfrac{\partial^2 V}{\partial x^2} + r x \dfrac{\partial V}{\partial x}
       - r V = 0, \qquad \tau \in (0, T], x \in (0, x_{\tmop{MAX}})\\
       V (x, T) = \max (E - x, 0),\\
       V (0, \tau) = E e^{- r (T - \tau)}, \quad V (x_{\tmop{MAX}}, \tau) = 0.
     \end{array}\right. \]
  In the following, set $E = 30$, $r = 0.05$, $\sigma = 0.1$, $x_{\tmop{MAX}}
  = 100$, $T = 1.$ This is a final value problem. In the HandsOn session we
  converted it into a classical initial value problem with the change of
  variable $t = T - \tau$, so that the final time $\tau = T$ is mapped into
  the initial time $t = 0$.
  \begin{enumeratenumeric}
    \item Write the problem in the {\tmstrong{``divergence'' initial-value}}
    form (for $t = T - \tau$):
    \[ \begin{array}{l}
         \dfrac{\partial V}{\partial t} - \dfrac{\partial}{\partial x} \left(
         \mu (x) \dfrac{\partial^{} V}{\partial x^{}} \right) + \beta (x)
         \dfrac{\partial V}{\partial x} + \sigma V = 0, \qquad t \in (0, T], x
         \in (0, x_{\tmop{MAX}})\\
         V (x, 0) = \max (E - x, 0),\\
         V (0, t) = E e^{- r (T - \tau)}, \quad V (x_{\tmop{MAX}}, t) = 0
       \end{array}, \]
    by finding appropriate coefficents $\mu (x) > 0$, $\beta (x)$ and $\sigma
    .$
    
    \item Write the weak formulation of the problem and its Finite Element
    approximation with the $\theta$-method. Specify the expected convergence
    rates.
    
    \item Solve the finite element approximation and compare the results with
    the finite difference solution found in HandsOn session Week 7. Try with
    different discretization parameters $h$ and $\Delta t$. (Suggested initial
    values: $h = 1$, $\Delta t = 0.01$).
    
    \item Report a plot of the solution $V (x, \tau = 0)$. In particular,
    compute {\tmname{$V (32, \tau = 0)$}} knowing that the ``exact'' solution
    reads 0.1892. Comment on the accuracy of your solver as a function of $h$
    and $\Delta t$.
  \end{enumeratenumeric}
  
\end{problem}

\begin{problem}
  Let us consider now the 2D Black and Scholes equation Put Option, where the
  option with value $V$ is based on two assets with price $x$ and $y$. The
  equation reads for $\tau \in (0, T] :$
  \[ \dfrac{\partial V}{\partial \tau} + \dfrac{1}{2} \sigma_1^2 x^2
     \dfrac{\partial^2 V}{\partial x^2} + \dfrac{1}{2} \sigma_2^2 y^2
     \dfrac{\partial^2 V}{\partial y^2} + \rho \sigma_1 \sigma_2 x y
     \dfrac{\partial^2 V}{\partial x \partial y} + r x \dfrac{\partial
     V}{\partial x} + r y \dfrac{\partial V}{\partial y} - r V = 0 \quad
     \tmop{with} (x, y) \in [0, x_{\tmop{MAX}}] \times [0, y_{\tmop{MAX}}] \]
  with the final condition $V (x, y, T) = \max (0, E - (x + y))$ and the
  boundary conditions
  \[ \left\{\begin{array}{l}
       V (x, 0, \tau) = g (x, \tau)\\
       V (x, y_{\tmop{MAX}}, \tau) = 0\\
       V (0, y, \tau) = g (y, \tau)\\
       V (x_{\tmop{MAX}}, y, \tau) = 0
     \end{array}\right. \]
  where $g (x, \tau)$ (or $g (y, \tau)$) is the solution of the 1D Black and
  Scholes equation (when one of the asset is worthless, the problem reduces to
  the problem only for the other asset). Here, $\rho$ is the correlation
  coefficient between the two assets. Data: $E = 40$, $r = 0.1$, $\sigma_1 =
  0.1$, $\sigma_2 = 0.3,$ $x_{\tmop{MAX}} = y_{\tmop{MAX}} = 100$, $T = 0.5,
  \rho = 0.5$.
\end{problem}

{\tmstrong{PART 1: Finite Differences}}. Rewrite the problem as an initial
value problem with the change of variable $t = T - \tau$. Write a code for the
finite difference solution of the previous problem, by combining the 1D solver
used in HandsOn7 for the boundary conditions $g$ with a 2D solver. For the
discretization of the mixed derivative, you can refer to the following scheme:
\[ \dfrac{\partial^2 V}{\partial x \partial y} (x_{i,} y_j) \approx
   \dfrac{1}{2 \Delta x} \left( \dfrac{\partial V}{\partial y} (x_{i + 1},
   y_j) - \dfrac{\partial V}{\partial y} (x_{i - 1}, y_j) \right) \approx
   \dfrac{1}{4 \Delta x \Delta y} (V_{i + 1, j + 1} - V_{i + 1, j - 1} - V_{i
   - 1, j + 1} + V_{i - 1, j - 1}) . \]
Taking the value $V (18, 20, \tau = 0)$=2.0187 as reference value, compare the
accuracy of your solver for different values of the discretization parameters
(suggested values to start with: $\Delta x = \Delta y = 1$ and $\Delta t =
0.01$).

{\tmem{BONUS: This is not necessary. This bonus is worth 10 points. Recalling
the Taylor expansion in 2D
\[ f (x + \Delta x, y + \Delta y) = f (x, y) + \dfrac{\partial f}{\partial x}
   (x, y) \Delta x + \dfrac{\partial f}{\partial y} (x, y) \Delta y +
   \dfrac{1}{2!} \left[ \dfrac{\partial^2 f}{\partial x^2} (x, y) \Delta x^2 +
   2 \dfrac{\partial^2 f}{\partial x \partial y} (x, y) \Delta x \Delta y +
   \dfrac{\partial^2 f}{\partial y^2} (x, y) \Delta y^2 \right] + \sum_{n =
   3}^{\infty} \dfrac{1}{n!} \sum^n_{k = 0} \left( \begin{array}{c}
     n\\
     k
   \end{array} \right) \dfrac{\partial^n f}{\partial x^k \partial y^{n - k}}
   \Delta x^k \Delta y^{n - k}, \]
for $h = \Delta x = \Delta y$, find the convergence rate $p$ of the previous
formula for the mixed second derivative (i.e. the error as $\mathcal{O}
(h^p)$).

\ }}

{\tmstrong{PART 2: Finite Elements. }}

1) Verify that the problem admits the following ``divergence'' initial value
form, for $t = T - \tau$:
\[ \dfrac{\partial V}{\partial t} - \nabla \cdot (\mathbb{T} \nabla V)
   +\tmmathbf{\beta} \cdot \nabla V + r V = 0 \]
where
\[ \mathbb{T} \equiv \dfrac{1}{2} \left(\begin{array}{cc}
     \sigma_1^2 x^2 & \rho \sigma_1 \sigma_2 x y\\
     \rho \sigma_1 \sigma_2 x y & \sigma^2_2 y^2
   \end{array}\right) \]
is a matrix (or tensor) and
\[ \tmmathbf{\beta} \equiv \left( \begin{array}{c}
     \left( \sigma_1^2 + \dfrac{1}{2} \rho \sigma_1 \sigma_2 - r \right) x\\
     \left( \sigma_2^2 + \dfrac{1}{2} \rho \sigma_1 \sigma_2 - r \right) y
   \end{array} \right) \]
is a vector.

Note that here $\mathbb{T} \nabla V$ denotes the matrix-vector product
\[ \dfrac{1}{2} \left(\begin{array}{cc}
     \sigma_1^2 x^2 & \rho \sigma_1 \sigma_2 x y\\
     \rho \sigma_1 \sigma_2 x y & \sigma^2_2 y^2
   \end{array}\right) \left( \begin{array}{c}
     \dfrac{\partial V}{\partial x}\\
     \dfrac{\partial V}{\partial y}
   \end{array} \right) = \dfrac{1}{2} \left( \begin{array}{c}
     \sigma^2_1 x^2 \dfrac{\partial V}{\partial x} + \rho \sigma_1 \sigma_2 x
     y \dfrac{\partial V}{\partial y}\\
     \rho \sigma_1 \sigma_2 x y \dfrac{\partial V}{\partial x} + \sigma^2_2
     y^2 \dfrac{\partial V}{\partial y}
   \end{array} \right) \]
and that $\nabla \cdot (\mathbb{T} \nabla V)$ stands for the divergence of the
$\mathbb{T} \nabla V$ vector:
\[ \nabla \cdot (\mathbb{T} \nabla V) \equiv \dfrac{\partial}{\partial x}
   \left( \dfrac{1}{2} \left( \sigma^2_1 x^2 \dfrac{\partial V}{\partial x} +
   \rho \sigma_1 \sigma_2 x y \dfrac{\partial V}{\partial y} \right) \right) +
   \dfrac{\partial}{\partial y} \left( \dfrac{1}{2} \left( \rho \sigma_1
   \sigma_2 x y \dfrac{\partial V}{\partial x} + \sigma^2_2 y^2
   \dfrac{\partial V}{\partial y} \right) \right) . \]
2) Write the weak formulation of the problem in this divergence form. To this
aim, keep in mind that for a test function $\varphi (x, y)$ vanishing on the
boundary
\[ - \int_{\Omega} \nabla \cdot (\mathbb{T} \nabla V) \varphi d \omega =
   \int_{\Omega} \mathbb{T} \nabla V \cdot \nabla \varphi d \omega \]
where
\[ \mathbb{T} \nabla V \cdot \nabla \varphi \equiv \dfrac{1}{2} \left(
   \begin{array}{c}
     \sigma^2_1 x^2 \dfrac{\partial V}{\partial x} + \rho \sigma_1 \sigma_2 x
     y \dfrac{\partial V}{\partial y}\\
     \rho \sigma_1 \sigma_2 x y \dfrac{\partial V}{\partial x} + \sigma^2_2
     y^2 \dfrac{\partial V}{\partial y}
   \end{array} \right) \cdot \left( \begin{array}{c}
     \dfrac{\partial \varphi}{\partial x}\\
     \dfrac{\partial \varphi}{\partial y}
   \end{array} \right) = \dfrac{1}{2} \left( \left( \sigma^2_1 x^2
   \dfrac{\partial V}{\partial x} + \rho \sigma_1 \sigma_2 x y \dfrac{\partial
   V}{\partial y} \right) \dfrac{\partial \varphi}{\partial x} + \left( \rho
   \sigma_1 \sigma_2 x y \dfrac{\partial V}{\partial x} + \sigma^2_2 y^2
   \dfrac{\partial V}{\partial y} \right) \dfrac{\partial \varphi}{\partial y}
   \right) . \]


3) Write the finite element+$\theta -$method discretization of the problem.
Specify the expected accuracy of the discretization, assuming that the
solution is infinitely regular.

{\tmstrong{{\tmem{BONUS - This part is not necessary. It is worth 40 bonus
points.
Write a Fenics code for the problem. To do so, consider that the function
{\tmtt{inner}} can perform the needed products. Alternatively, the derivative
$\dfrac{\partial V}{\partial x}$ can be written as {\tmtt{V.dx(0)}} while
$\dfrac{\partial V}{\partial y}$ is {\tmtt{V.dx(1)}} (and similar for
$\varphi$). Compare the solution with the finite difference one.}}}}



\end{document}
